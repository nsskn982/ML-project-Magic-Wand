%%%%%%
%
% $Autor: Sudeshna Nanda $
% $Date: 2025-01-15 $
% $File name:  ProjectPresentation.tex
% $Version: 1.0 $
%
% !TeX encoding = utf8
%
%%%%%%


\Mysection{Challenges}

\begin{frame}{Challenge 1}

\textbf{Resource Constraints}
	
	\begin{itemize}
		
		\item The limited memory and computational power of the Arduino Nano 33 BLE Sense required optimization techniques to adapt machine learning models for deployment. [\cite{Ard:2021}].
		
		\item High-performance deep learning models designed for GPUs or clusters are incompatible with microcontroller units (MCUs), demanding significant model simplifications.
		
	\end{itemize}
	
\end{frame}
	
\begin{frame}{Challenges 2}
\textbf{Gesture Recognition Complexity}

	\begin{itemize}
	
		\item Developing a robust ML model to recognize predefined gestures (wing, ring, slope).
	
		\item Accurately distinguishing untrained or "unknown" gestures.
	
		\item Avoiding false positives caused by overlapping or extended gestures, which could mislead the accelerometer readings.
	
	\end{itemize}
	
\end{frame}

\begin{frame}{Challenges 3}
\textbf{Model Optimization}
	
	\begin{itemize}

		\item Ensuring real-time performance while keeping model size and computational complexity low – a critical balance for edge devices.
	
		\item Employing techniques like quantization and pruning to minimize resource usage without degrading accuracy.

	\end{itemize}
\end{frame}
	
\begin{frame}{Challenges 4}
\textbf{Framework Adaptation}

	\begin{itemize}
	
		\item Leveraging tools like TensorFlow Lite and TinyML frameworks, which are not fully optimized for intricate applications.
	
		\item Tailoring frameworks to meet project-specific requirements.
\end{itemize}

\end{frame}
