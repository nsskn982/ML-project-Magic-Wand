%%%%%%
%
% $Autor: Adhiraj Walse $
% $Date: 2025-01-15 $
% $File name:  ProjectPresentation.tex
% $Version: 1.0 $
%
% !TeX encoding = utf8
%
%%%%%%

\Mysection{Solutions}
\begin{frame}{Solution 1: Optimization Techniques for MCUs}
	\begin{itemize}
		\item Use \textbf{model quantization} to convert 32-bit floating point models into 8-bit integer models[\cite{Ard:2021}].
		\item Implement \textbf{pruning techniques} to remove insignificant model weights.
		\item Leverage \textbf{TinyML frameworks}, such as TensorFlow Lite for Microcontrollers.
	\end{itemize}
\end{frame}


\begin{frame}{Solution 2: Enhanced Data Collection and Model Training}
	\begin{itemize}
		\item Gather a diverse and high-quality dataset covering predefined and random gestures[\cite{shi:2016}].
		\item Incorporate \textbf{data augmentation} to simulate real-world noise and variations.
		\item Use \textbf{1-D Convolutional Neural Networks (CNNs)} for effective feature extraction.
	\end{itemize}
\end{frame}


\begin{frame}{Solution 3: Balancing Accuracy and Efficiency}
	\begin{itemize}
		\item Optimize the neural network architecture by reducing layers or parameters[\cite{shi:2016}].
		\item Apply \textbf{transfer learning} to adapt pre-trained models and reduce training time.
		\item Utilize profiling tools for performance benchmarking and fine-tuning.
	\end{itemize}
\end{frame}


\begin{frame}{Solution 4: Customizing Frameworks}
	\begin{itemize}
		\item Extend TensorFlow Lite for Microcontrollers with project-specific optimizations[\cite{War:2020}].
		\item Use \textbf{hardware-specific libraries} like the Arduino TensorFlow Lite library.
		\item Implement modular testing for incremental verification.
	\end{itemize}
\end{frame}

