%%%%%%%%%%%%
%
% $Autor: Sudeshna,Srikanth Nanda, Adhiraj $
% $Datum: 2025-01-14 08:03:15Z $
% $Pfad: TemplateSensor $
% $Version: 4250 $
% !TeX spellcheck = en_GB/de_DE
% !TeX encoding = utf8
% !TeX root = filename 
% !TeX TXS-program:bibliography = txs:///biber
%
%%%%%%%%%%%%


\chapter{Development to Development}
\label{chapter 8}

\section{Development to Deployment}

\subsection{1. Development Phase}

\subsubsection{Define Problem}
\begin{itemize}
    \item Recognize gestures (e.g., "wing," "ring," "slope") using an accelerometer.
    \item Minimize model size due to memory constraints of the Arduino Nano 33 BLE Sense.
\end{itemize}

\subsubsection{Data Collection}
\begin{itemize}
    \item Collect gesture data using the board's sensors (accelerometer, gyroscope).
    \item Label data corresponding to gestures, including "unknown" gestures.
\end{itemize}

\subsubsection{Data Preprocessing}
\begin{itemize}
    \item Filter out noise using smoothing algorithms.
    \item Normalize accelerometer values to ensure consistency across datasets.
\end{itemize}

\subsubsection{Model Design}
\begin{itemize}
    \item Use TensorFlow to design a CNN model for gesture recognition.
    \item Choose lightweight architectures suitable for TinyML.
\end{itemize}

\subsubsection{Model Training}
\begin{itemize}
    \item Split the dataset into training, validation, and test sets.
    \item Optimize the model using quantization, pruning, or knowledge distillation to fit edge device constraints.
\end{itemize}

\subsection{2. Tools and Libraries}

\subsection{1. Hardware}

\subsubsection{Arduino Nano 33 BLE Sense}
\begin{itemize}
    \item Sensors: Accelerometer, gyroscope, magnetometer, temperature, pressure.
    \item Features: Bluetooth Low Energy (BLE) for communication, small form factor.
    \item Use Case: Collect gesture data, run TinyML models, and communicate results via BLE.
    \item \textbf{Documentation Link}: \url{https://www.arduino.cc/en/Guide/NANO33BLESense}
\end{itemize}

\subsubsection{Peripheral Tools}
\begin{itemize}
    \item USB Cable: For data transfer and power supply.
    \item Sticky Tape: To secure the Arduino to the wand (physical setup).
    \item External LEDs or buzzers (optional): For additional feedback mechanisms.
\end{itemize}

\subsection{2. Development Environment}

\subsubsection{Arduino IDE}
\begin{itemize}
    \item Version: Latest stable release.
    \item Use Case: Write and upload C++ code to the Arduino Nano 33 BLE Sense.
    \item Extensions: Add board-specific libraries like \texttt{Arduino\_TensorFlowLite}.
    \item Installation: Available for Windows, Mac, and Linux.
    \item \textbf{Download Link}: \url{https://www.arduino.cc/en/software}
\end{itemize}

\subsubsection{Python}
\begin{itemize}
    \item Version: 3.8 or later.
    \item Use Case: Data preprocessing, model development, and testing.
    \item Popular IDEs: PyCharm, Visual Studio Code, or Jupyter Notebook.
\end{itemize}

\subsection{3. Libraries for Data Preprocessing}
\begin{itemize}
    \item \textbf{NumPy}: 
    \begin{itemize}
        \item Use Case: Perform numerical operations on accelerometer data, such as normalization and smoothing.
        \item Installation: \texttt{pip install numpy}
    \end{itemize}
    \item \textbf{Pandas}: 
    \begin{itemize}
        \item Use Case: Organize sensor data into data frames for easy manipulation and analysis.
        \item Installation: \texttt{pip install pandas}
    \end{itemize}
    \item \textbf{Matplotlib}: 
    \begin{itemize}
        \item Use Case: Visualize gesture patterns (e.g., acceleration over time).
        \item Installation: \texttt{pip install matplotlib}
    \end{itemize}
\end{itemize}

\subsection{4. Machine Learning Libraries}
\begin{itemize}
    \item \textbf{TensorFlow}: 
    \begin{itemize}
        \item Version: TensorFlow 2.x.
        \item Use Case: Model creation, training, and optimization.
        \item Installation: \texttt{pip install tensorflow}
    \end{itemize}
    \item \textbf{TensorFlow Lite Converter}: 
    \begin{itemize}
        \item Use Case: Convert TensorFlow models to a format suitable for microcontrollers.
        \item Integrated with TensorFlow.
    \end{itemize}
    \item \textbf{TensorFlow Lite for Microcontrollers}: 
    \begin{itemize}
        \item Use Case: Load and run the \texttt{.tflite} model on the Arduino Nano 33 BLE Sense.
        \item Integrated into Arduino libraries.
    \end{itemize}
\end{itemize}

\subsection{5. Deployment Libraries}
\begin{itemize}
    \item \textbf{Arduino\_TensorFlowLite}: 
    \begin{itemize}
        \item Use Case: Interface TensorFlow Lite models with the Arduino Nano 33 BLE Sense.
        \item Installation: Install via Arduino Library Manager.
    \end{itemize}
    \item \textbf{Arduino\_LSM9DS1}: 
    \begin{itemize}
        \item Use Case: Access accelerometer, gyroscope, and magnetometer data.
        \item Installation: Available in the Arduino IDE.
    \end{itemize}
\end{itemize}

\subsection{6. Optional Libraries}
\begin{itemize}
    \item \textbf{Scikit-learn}: 
    \begin{itemize}
        \item Use Case: Preprocessing tools like scaling, splitting datasets, and evaluating performance metrics.
        \item Installation: \texttt{pip install scikit-learn}
    \end{itemize}
    \item \textbf{Seaborn}: 
    \begin{itemize}
        \item Use Case: Enhanced data visualizations for exploring gesture patterns.
        \item Installation: \texttt{pip install seaborn}
    \end{itemize}
    \item \textbf{Keras}: 
    \begin{itemize}
        \item Use Case: High-level API for rapid prototyping of the CNN model.
        \item Installation: \texttt{pip install keras}
    \end{itemize}
\end{itemize}

\subsection{7. Version Control and Collaboration}
\begin{itemize}
    \item \textbf{Git}: 
    \begin{itemize}
        \item Use Case: Track code changes and collaborate with team members.
        \item Integration: Platforms like GitHub or GitLab.
    \end{itemize}
    \item \textbf{Google Colab (Optional)}: 
    \begin{itemize}
        \item Use Case: Train and test models in a cloud-based Jupyter environment.
    \end{itemize}
\end{itemize}

\subsection{8. Debugging and Profiling Tools}
\begin{itemize}
    \item \textbf{Serial Monitor (Arduino IDE)}: 
    \begin{itemize}
        \item Use Case: Debug real-time sensor readings and verify program behavior.
    \end{itemize}
    \item \textbf{TensorBoard}: 
    \begin{itemize}
        \item Use Case: Monitor model training metrics like loss and accuracy.
        \item Command to launch: \texttt{tensorboard --logdir=logs/}
    \end{itemize}
\end{itemize}

\subsection{3. File Structure}

Here's the directory structure in LaTeX using plain text and removing Unicode symbols:

latex
Copy code
\begin{verbatim}
	MagicWandProject/
	|-- data/
	|   |-- raw/                # Raw sensor data
	|   |-- processed/          # Preprocessed data
	|   |-- labels.csv          # Gesture labels
	|-- models/
	|   |-- cnn_model.h5        # Saved Keras model
	|   |-- cnn_model.tflite    # TensorFlow Lite model
	|-- src/
	|   |-- preprocess.py       # Data preprocessing scripts
	|   |-- train.py            # Model training script
	|   |-- deploy/
	|       |-- main.ino        # Arduino C++ script
	|       |-- tflite_integration.ino  # TFLite integration code
	|-- requirements.txt        # Required Python libraries
	|-- README.md               # Project description
\end{verbatim}

\subsection{Saving and Loading Models}

\subsubsection{Saving the Model}
The following Python code demonstrates how to save a model in the HDF5 format and convert it to TensorFlow Lite using `TFLiteConverter`.

\begin{lstlisting}[language=Python, caption={Saving and Converting a Model to TensorFlow Lite}, label={code:tf-lite-conversion}, style=pythonstyle]
	import tensorflow as tf
	
	# Save the model in HDF5 format
	model.save('models/cnn_model.h5')
	
	# Convert to TensorFlow Lite format
	converter = tf.lite.TFLiteConverter.from_saved_model('models/cnn_model.h5')
	tflite_model = converter.convert()
	
	# Save the converted model
	with open('models/cnn_model.tflite', 'wb') as f:
	f.write(tflite_model)
\end{lstlisting}


\subsubsection{Loading the Model}
\begin{lstlisting}[language=Python, caption={Loading a Saved Model for Evaluation}, label={code:tf-load-model}, style=pythonstyle]
	import tensorflow as tf
	
	# Load the model for evaluation
	model = tf.keras.models.load_model('models/cnn_model.h5')
\end{lstlisting}

