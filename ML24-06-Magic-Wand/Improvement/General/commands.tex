%%%%%%
%
% $Autor: Wings $
% $Datum: 2020-01-18 11:15:45Z $
% $Pfad: WuSt/Skript/Produktspezifikation/powerpoint/ImageProcessing.tex $
% $Version: 4620 $
%
%%%%%%



\graphicspath{%
	{../Images/},{../../Images/},{Images/},%
	{../../../../General/},{../../../General/},{../../General/},%
	{../../../../Aufgaben/},{../../../Aufgaben/},%
	{../}}


\definecolor{MapleColor}{rgb}{1,0.0,0.}
\definecolor{PythonColor}{rgb}{0,0.5,1.}
\definecolor{ShellColor}{rgb}{1,0,0.5}
\definecolor{FileColor}{rgb}{0.5,0.5,1.}

\definecolor{orange}{RGB}{255,200,0}
\definecolor{LightBlue}{RGB}{173, 216, 250}
\definecolor{Blue}{RGB}{50, 100, 250}
\definecolor{BurntOrange}{RGB}{250,180,30}
\definecolor{DarkGreen}{RGB}{50,150,50}
\definecolor{White}{RGB}{255,255,255}
\definecolor{Gray}{RGB}{128,128,128}
\definecolor{LightGrey}{RGB}{220,220,220}
\definecolor{VLightGrey}{RGB}{245,245,245}


\newcommand{\MapleCommand}[1]{\textcolor{MapleColor}{\texttt{\justify#1}}}
\newcommand{\PYTHON}[1]{\textcolor{PythonColor}{\texttt{\justify#1}}}
\newcommand{\SHELL}[1]{\textcolor{ShellColor}{\texttt{\justify#1}}}
\newcommand{\FILE}[1]{\textcolor{FileColor}{\texttt{\justify#1}}\index{\TRANS{Datei}{File}!#1}}
\newcommand{\PATH}[1]{\textcolor{FileColor}{\texttt{\justify#1}}}

\newcommand{\URL}[1]{\textcolor{blue}{\url{#1}}}
\newcommand{\HREF}[2]{\textcolor{blue}{\href{#1}{#2}}}


\newcommand{\QUELLE}{\textcolor{red}{hier Quelle finden}}

\newcolumntype{L}[1]{>{\raggedright\arraybackslash}p{#1}} % linksbündig mit Breitenangabe

\newcommand{\GRAFIK}{\textcolor{red}{Grafik einfügen}}

\newcommand{\DEF}[1]{\fcolorbox{blue}{blue!10}{\begin{minipage}{\textwidth}\textbf{Definition.}#1\end{minipage}}}

\newcommand{\BEISPIEL}[1]{\fcolorbox{blue}{blue!10}{\begin{minipage}{\textwidth}\textbf{Beispiel.}#1\end{minipage}}}

\newcommand{\SATZ}[1]{\fcolorbox{blue}{blue!10}{\begin{minipage}{\textwidth}\textbf{Satz.}#1\end{minipage}}}

\newcommand{\Bemerkung}[1]{\fcolorbox{blue}{blue!10}{\begin{minipage}{\textwidth}\textbf{Bemerkung.}#1\end{minipage}}}


%Zahlenmengen
\newcommand{\C}{\mathbb{C}}
\newcommand{\R}{\mathbb{R}}
\newcommand{\N}{\mathbb{N}}
\newcommand{\Z}{\mathbb{Z}}
\newcommand{\Q}{\mathbb{Q}}
\newcommand{\Po}{\mathbb{P}}
\newcommand{\Rp}[2]{\mathbb{R}^{[#1;\,#2]}} % Menge der auf [a,b]-periodischen Funktionrn
%\DeclareMathOperator{\arg}{arg}
\newcommand{\MyComplex}[1]{\mathbf{#1}}
\newcommand{\Laplaceinv}[1]{#1}
\newcommand{\Bez}{Bézier}
\newcommand{\PH}{pythagoreische Hodographen \xspace}
\newcommand{\PHd}{pythagoreischen Hodographen \xspace}


\DeclareMathOperator{\Atan2}{Atan2}
\DeclareMathOperator{\sign}{sign}
\DeclareMathOperator{\ReLU}{ReLU}

\newcommand*\justify{%
	\fontdimen2\font=0.4em% interword space
	\fontdimen3\font=0.2em% interword stretch
	\fontdimen4\font=0.1em% interword shrink
	\fontdimen7\font=0.1em% extra space
	\hyphenchar\font=`\-% allowing hyphenation
}

% Auswahl der Sprache
% 1.Argument ist der Pfad ohne "en" oder "de"
% 2.Argument ist der Dateiname
\newcommand{\InputLanguage}[2]{
    \ifdefined\isGerman
    \input{../../#1de/#2}
    \else
    \ifdefined\isEnglish
    \input{../../#1en/#2}
    \else
    \input{../../#1de/#2}
    \fi
    \fi
}

\newcommand{\TRANS}[2]{
    \ifdefined\isGerman
    #1%
    \else
    \ifdefined\isEnglish
    #2%
    \else
    #1%
    \fi
    \fi
}


% muss für Akronyme \ac statt see verwendet werden.
\newcommand{\Siehe}{
    \ifdefined\isGerman
    \emph{siehe}
    \else
    \ifdefined\isEnglish
    \emph{see}
    \else
    \emph{siehe}
    \fi
    \fi
}

\renewcommand{\indexname}{
	\ifdefined\isGerman
	\emph{Stichwortverzeichnis}
	\else
	\ifdefined\isEnglish
	\emph{Index}
	\else
	\emph{Stichwortverzeichnis}
	\fi
	\fi
}


%todo Die Kommandos sind für das Endprodukt zu entfernen. Die entsprechenden Stellen sind zu bearbeiten bzw. zu löschen
\newcommand{\Mynote}[1]{\marginnote{\textcolor{red}{WS:#1}}}
\newcommand{\Ausblenden}[1]{}
\newcommand{\ToDo}[1]{\textcolor{red}{\section{ToDo} #1}}

\newcommand{\GRAFIKEINFUEGEN}{\textcolor{red}%
	{\bf Hier ist eine Grafik einzufügen}}
\newcommand{\neu}{\textcolor{red}{Wort Einfügen}}
\newcommand{\REFERENZ}{\textcolor{red}%
	{\bf Referenz  einzufügen}}
\newcommand{\EXAMPLE}{\textcolor{red}%
	{\bf Beispiel einzufügen}}

\newcommand{\trinom}[3]{\left(\begin{array}{c} #1\\#2\\#3 \\ \end{array}\right)}


\newcommand{\textdirectcurrent}{%
	\settowidth{\dimen0}{$=$}%
	\vbox to .85ex {\offinterlineskip
		\hbox to \dimen0{\leaders\hrule\hfill}
		\vskip.35ex
		\hbox to \dimen0{%
			\leaders\hrule\hskip.2\dimen0\hfill
			\leaders\hrule\hskip.2\dimen0\hfill
			\leaders\hrule\hskip.2\dimen0
		}
		\vfill
	}%
}
\newcommand{\mathdirectcurrent}{\mathrel{\textdirectcurrent}}

\newlist{notes}{enumerate}{1}
\setlist[notes]{label=Note: ,leftmargin=*}

% Kommandos für In-Line Code
\newcommand{\COMMAND}[1]{\lstinline[language=MyBash, style=inlinestyle]{#1}}
\newcommand{\mytt}[1]{\texttt{\footnotesize #1}}
\newcommand{\python}[1]{\lstinline[language=MyPython, style=inlinestyle]{#1}}
\newcommand{\twod}[2]{
	\ensuremath{{#1} \times {#2}}}
\newcommand{\threed}[3]{
	\ensuremath{{#1} \times {#2} \times {#3}}}

% Fett-geschriebene Tabellenberschriften
\renewcommand\theadfont{\bfseries}


\newcommand{\us}{\si{\micro\second}}
\newcommand{\usn}[1]{\SI{#1}{\micro\second}}
\renewcommand{\textapprox}{\raisebox{0.5ex}{\texttildelow}}



% Fehlerzähler
\setul{0.5ex}{0.3ex}
\setulcolor{red}
\newcounter{fehlernummer}
\setcounter{fehlernummer}{11}
\newcommand{\FEHLER}[1]{\ul{#1}\stepcounter{fehlernummer}\textsuperscript{\textcolor{red}{\arabic{fehlernummer}}}}
%\renewcommand{\FEHLER}[1]{\ul{#1}\stepcounter{fehlernummer}\marginnote{\textcolor{red}{\arabic{fehlernummer}}}} geht leider nicht
%\renewcommand{\FEHLER}[1]{\ul{#1}\stepcounter{fehlernummer}\footnote{\textcolor{red}{\arabic{fehlernummer}.Fehler}}} geht leider nicht


\newcounter{FortlaufendeNummer}
\setcounter{FortlaufendeNummer}{1}
\newcounter{LetztesKapitel}
\setcounter{LetztesKapitel}{-1}

\newcounter{BeamerChapter}
\setcounter{BeamerChapter}{1}


\newcommand{\GRAPHICSC}[3]{\begin{center}
		\includegraphics[scale=#1]{#3}
\end{center}}

\newcommand{\GRAPHICS}[3]{\includegraphics[scale=#1]{#3}}

\newcommand{\FONO}{   
	%\ifthenelse{\value{chapter} > \value{LetztesKapitel}}%
	%{%
		%\setcounter{LetztesKapitel}{\value{chapter}}%
		%\setcounter{FortlaufendeNummer}{1}%
		%}%
	%\arabic{chapter}.\arabic{FortlaufendeNummer}.%
	%\stepcounter{FortlaufendeNummer}%
}

\newcommand{\FONOBEAMER}{%
	\ifthenelse{\value{BeamerChapter}>0}%
	{%
		\ifthenelse{\value{BeamerChapter}>\value{LetztesKapitel}}%
		{%
			\setcounter{LetztesKapitel}{\value{BeamerChapter}}%
			\setcounter{FortlaufendeNummer}{1}%
		}%
		\hbox{}\arabic{BeamerChapter}.\arabic{FortlaufendeNummer}.\stepcounter{FortlaufendeNummer}%
	}%
	\hbox{}%
} 


\newcommand{\FONOBEAMERSTEP}{%
	\ifthenelse{\value{BeamerChapter}>0}%
	{%
		\ifthenelse{\value{BeamerChapter}>\value{LetztesKapitel}}%
		{%
			\setcounter{LetztesKapitel}{\value{BeamerChapter}}%
			\setcounter{FortlaufendeNummer}{1}%
		}%
		\stepcounter{FortlaufendeNummer}%
	}%
	\hbox{}%
	\bigskip
} 

\newcommand{\FONOBEAMERSTAY}{   
	\ifthenelse{\value{BeamerChapter}>\value{LetztesKapitel}}
	{
		\setcounter{LetztesKapitel}{\value{BeamerChapter}}
		\setcounter{FortlaufendeNummer}{1}
	}
	\hbox{}\arabic{BeamerChapter}.\arabic{FortlaufendeNummer}.} 


% \FONOBEAMERSTEP
% \SATZNAME{}{}
% \SATZ{}
% \SATZNAMES{}{}
% \SATZS{}
% \STANDARD{}{}
% \STANDARDN{}{}  handout:0
% \STANDARDV{}{}  verbatim
% \DEF{}
% \DEFNAME{}{}
% \DEFS{}
% \DEFNAMES{}{}
% \BEMERKUNG{}
% \BEMERKUNGNAME{}{}
% \BEMERKUNGNAMES{}{}
% \BEISPIELNAME{}{}
% \BEISPIELNAMES{}{}
% \BEISPIEL{}
% \BEISPIELS{}



\newcommand{\Def}[1]
{
	\definecolor{shadecolor}{rgb}{0.98, 0.91, 0.71}
	
	\begin{snugshade*}
		\textbf{\FONOBEAMER  Definition.} #1
	\end{snugshade*}
	
	\bigskip
}

\newcommand{\DefN}[2]
{
	\definecolor{shadecolor}{rgb}{0.98, 0.91, 0.71}
	
	\begin{snugshade*}
		\textbf{ \FONOBEAMER Definition #1.}  #2
	\end{snugshade*}
	
	\bigskip
}


\newcommand{\Satz}[1]
{
	\definecolor{shadecolor}{rgb}{0.74, 0.83, 0.9}
	
	\begin{snugshade*}
		\textbf{\FONOBEAMER  Satz.} #1
	\end{snugshade*}
	
	\bigskip
}

\newcommand{\SatzN}[2]
{
	\definecolor{shadecolor}{rgb}{0.74, 0.83, 0.9}
	
	\begin{snugshade*}
		\textbf{ \FONOBEAMER Satz #1.}  #2
	\end{snugshade*}
	
	\bigskip
}


\newcommand{\BemerkungN}[2]
{
	\definecolor{shadecolor}{rgb}{0.66, 0.89, 0.63} %Lavendel
	
	\begin{snugshade*}
		\textbf{ \FONOBEAMER Bemerkung #1.}  #2
	\end{snugshade*}
	
	\bigskip
}


\newcommand{\Beispiel}[1]
{
	\definecolor{shadecolor}{rgb}{0.9, 0.9, 0.98} 
	
	\begin{snugshade*}
		\textbf{\FONOBEAMER  Beispiel.} #1
	\end{snugshade*}
	
	\bigskip
}

\newcommand{\BeispielN}[2]
{
	\definecolor{shadecolor}{rgb}{0.9, 0.9, 0.98} 
	
	\begin{snugshade*}
		\textbf{ \FONOBEAMER Beispiel #1.}  #2
	\end{snugshade*}
	
	\bigskip
}


\newcommand\pythonstyle{
	\lstset{ 
		backgroundcolor=\color{white},   % choose the background color; you must add \usepackage{color} or \usepackage{xcolor}; should come as last argument
		basicstyle=\footnotesize,        % the size of the fonts that are used for the code
		breakatwhitespace=false,         % sets if automatic breaks should only happen at whitespace
		breaklines=true,                 % sets automatic line breaking
		captionpos=b,                    % sets the caption-position to bottom
		commentstyle=\color{mygreen},    % comment style
		deletekeywords={...},            % if you want to delete keywords from the given language
		escapeinside={\%*}{*)},          % if you want to add LaTeX within your code
		extendedchars=true,              % lets you use non-ASCII characters; for 8-bits encodings only, does not work with UTF-8
		firstnumber=1000,                % start line enumeration with line 1000
		frame=single,	                   % adds a frame around the code
		keepspaces=true,                 % keeps spaces in text, useful for keeping indentation of code (possibly needs columns=flexible)
		keywordstyle=\color{blue},       % keyword style
		language=Python,                 % the language of the code
		morekeywords={*,...},            % if you want to add more keywords to the set
		numbers=left,                    % where to put the line-numbers; possible values are (none, left, right)
		numbersep=5pt,                   % how far the line-numbers are from the code
		numberstyle=\tiny\color{mygray}, % the style that is used for the line-numbers
		rulecolor=\color{black},         % if not set, the frame-color may be changed on line-breaks within not-black text (e.g. comments (green here))
		showspaces=false,                % show spaces everywhere adding particular underscores; it overrides 'showstringspaces'
		showstringspaces=false,          % underline spaces within strings only
		showtabs=false,                  % show tabs within strings adding particular underscores
		stepnumber=2,                    % the step between two line-numbers. If it's 1, each line will be numbered
		stringstyle=\color{mymauve},     % string literal style
		tabsize=2,	                     % sets default tabsize to 2 spaces
		title=\lstname                   % show the filename of files included with \lstinputlisting; also try caption instead of title
	} 
}

\lstset{language=Python}

% Python environment
\lstnewenvironment{Python}[1][]
{
	\pythonstyle
	\lstset{#1}
}
{}

%Python for external files
\newcommand\pythonexternal[2][]{{
		\pythonstyle
		\lstinputlisting[#1]{#2}}}

%Python for external files
\newcommand\PythonExternalO[1]{
    \arduinostyle
    \lstinputlisting[language=C++]{#1}}


%   Arduino-Programs
\newcommand\arduinostyle{
    \lstset{ 
        backgroundcolor=\color{white},   % choose the background color; you must add \usepackage{color} or \usepackage{xcolor}; should come as last argument
        basicstyle=\footnotesize,        % the size of the fonts that are used for the code
        breakatwhitespace=false,         % sets if automatic breaks should only happen at whitespace
        breaklines=true,                 % sets automatic line breaking
        captionpos=b,                    % sets the caption-position to bottom
        commentstyle=\color{mygreen},    % comment style
        deletekeywords={...},            % if you want to delete keywords from the given language
        escapeinside={\%*}{*)},          % if you want to add LaTeX within your code
        extendedchars=true,              % lets you use non-ASCII characters; for 8-bits encodings only, does not work with UTF-8
        firstnumber=1000,                % start line enumeration with line 1000
        frame=single,	                   % adds a frame around the code
        keepspaces=true,                 % keeps spaces in text, useful for keeping indentation of code (possibly needs columns=flexible)
        keywordstyle=\color{blue},       % keyword style
        language=C++,                    % the language of the code
        morekeywords={*,...},            % if you want to add more keywords to the set
        numbers=left,                    % where to put the line-numbers; possible values are (none, left, right)
        numbersep=5pt,                   % how far the line-numbers are from the code
        numberstyle=\tiny\color{mygray}, % the style that is used for the line-numbers
        rulecolor=\color{black},         % if not set, the frame-color may be changed on line-breaks within not-black text (e.g. comments (green here))
        showspaces=false,                % show spaces everywhere adding particular underscores; it overrides 'showstringspaces'
        showstringspaces=false,          % underline spaces within strings only
        showtabs=false,                  % show tabs within strings adding particular underscores
        stepnumber=2,                    % the step between two line-numbers. If it's 1, each line will be numbered
        stringstyle=\color{mymauve},     % string literal style
        tabsize=2,	                     % sets default tabsize to 2 spaces
        title=\lstname                   % show the filename of files included with \lstinputlisting; also try caption instead of title
    } 
}

\lstset{language=C++}

% Ardunio environment
\lstnewenvironment{Arduino}[1][]
{
    \arduinostyle
    \lstset{#1}
}
{}

%Arduino for external files
\newcommand\ArduinoExternal[2]{{
        \arduinostyle
        \lstinputlisting[#1]{#2}}}

%Arduino for external files
\newcommand\ArduinoExternalO[1]{
        \arduinostyle
        \lstinputlisting[language=C++]{#1}}



\setlength{\parindent}{0pt}




\definecolor{uuuuuu}{rgb}{0.26666666666666666,0.2666666666666666,0.26666666666666666}
\definecolor{qqqqff}{rgb}{0.,0.,1.}
\definecolor{MapleColor}{rgb}{1,0.5,0.}
\definecolor{PythonColor}{rgb}{0,0.5,1.}
\definecolor{ShellColor}{rgb}{1,0,0.5}
\definecolor{FileColor}{rgb}{0.5,0.5,1.}

\definecolor{LightGoldenrod}{rgb}{0.8,.9,0.3} 
\definecolor{AliceBlue}{rgb}{0.5,.8,1}
\definecolor{LightGrey}{rgb}{0.9,0.9,0.9} 
\definecolor{Beige}{rgb}{0.9,0.5,0.0} 
\definecolor{Gelb}{rgb}{0.999,0.999,0.0} 


\definecolor{LightCyan}{rgb}{0.88,1,1}
\definecolor{frenchblue}{rgb}{0.0, 0.45, 0.73}
\definecolor{greenblue}{rgb}{0.0, 0.25, 0.3}
\definecolor{darkcyan}{rgb}{0.0, 0.55, 0.55}
\definecolor{bondiblue}{rgb}{0.0, 0.58, 0.71}
\definecolor{grayleft}{rgb}{0.1, 0.1, 0.1}
\definecolor{grayright}{rgb}{0.2, 0.2, 0.2}
\definecolor{graycircle}{rgb}{0.3, 0.3, 0.3}
\definecolor{graylight}{rgb}{0.8, 0.8, 0.8}
\definecolor{greenenglish}{rgb}{0.0, 0.5, 0.0}
\definecolor{darkpastelgreen}{rgb}{0.01, 0.75, 0.24}
\definecolor{copper}{rgb}{0.72, 0.45, 0.2}
\definecolor{greenyellow}{rgb}{0.68, 1.0, 0.18}
\definecolor{fuchsia}{rgb}{1.0, 0.0, 1.0}
\definecolor{silver}{rgb}{0.75, 0.75, 0.75}
\definecolor{deepskyblue}{rgb}{0.0, 0.75, 1.0}


\newcommand{\GND}{\cellcolor{black}\textcolor{white}{GND}}
\newcommand{\Vf}{\cellcolor{red}\textcolor{black}{5V}}
\newcommand{\Vd}{\cellcolor{red}\textcolor{black}{3.3V}}
\definecolor{LightCyan}{rgb}{0.88,1,1}

\newcolumntype{a}{>{\columncolor{LightCyan}}c}

\DeclareCaptionType{code}[Listing][\TRANS{Liste des Listings}{List of Listings}] 
