%%%%%%%%%%%%%%%
%
% $Autor: Wings $
% $Datum: 2020-01-29 07:55:27Z $
% $Pfad: General/I2C.tex
% $Version: 1785 $
%
%
%%%%%%%%%%%%%%%

\chapter{Serial Peripheral Interface}

Das Akronym SPI steht für Serial Peripheral Interface. Dieses Protokoll wird von Mikrocontrollern genutzt, um eine schnelle Kommunikation mit einem oder mehreren Peripheriegeräten zu ermöglichen. Es gibt drei gemeinsame Pins für alle Peripheriegeräte:

\begin{itemize}
  \item SCK: Dies steht für Serial Clock und erzeugt Taktimpulse zur Synchronisation der Datenübertragung.
  \item  MISO: Dies steht für Master Input/Slave Output und wird genutzt, um Daten an den Master zu senden.
  \item MOSI: Dies steht für Master Output/Slave Input und wird genutzt, um Daten an die Slaves/Peripheriegeräte zu senden.
\end{itemize}

Die SPI-Pins auf der Platine sind D13 (SCK), D12 (MISO) und D11 (MOSI). RXD und TXD werden für die serielle Kommunikation genutzt.
Der TXD-Pin wird zur Übertragung von Daten genutzt. Die RXD-Pin für den Empfang von Daten, während der seriellen Kommunikation. Die Pins stellen auch den erfolgreichen Datenfluss vom Computer zur Platine her. Die UART-Pins auf der Platine sind D0 (TX) und D1 (RX).