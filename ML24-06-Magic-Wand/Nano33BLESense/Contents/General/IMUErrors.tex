%%%
%
% $Autor: Wings $
% $Datum: 2023-03-01 $
% $Dateiname: IMUErrors $
% $Version: 1 $
%
%%%


\chapter{Errors}  \label{Chap: Errors}

\section{Introduction}

IMU (Inertial Measurement Unit) is a sensor that measures and reports the linear and rotational motion of an object. The error in IMU refers to any deviation or inaccuracy in the measurements reported by the sensor from the true or expected values.

\bigskip


\begin{enumerate}
    \item\textbf{Bias}: Bias is the constant offset in the readings of the IMU. It can be caused by a variety of factors, including manufacturing defects, aging of the sensors, or changes in temperature. These errors can lead to systematic errors that can persist over time and can be difficult to detect. To correct for the bias error, the IMU must be calibrated periodically. Calibration involves comparing the IMU's measurements with a known reference, and then estimating and subtracting the bias from the measurements to obtain more accurate results.\cite{sabatini:2011}

    \item \textbf{Drift}: Drift refers to the gradual change in bias over time. It can be caused by aging of the sensors, temperature changes, or electronic interference. Unlike the bias error, drift can vary over time, and it can be challenging to detect and correct. As a result, it is essential to calibrate the IMU regularly to correct for drift. More advanced techniques such as Kalman filtering can also be used to estimate and compensate for drift over time.\cite{Tedaldi:2014}

    \item \textbf{Noise:} Noise refers to the random variations in the sensor readings that can affect the accuracy of the measurement. This error is especially pronounced when the IMU is stationary. The noise can be caused by various factors such as electronic interference, vibrations, or environmental conditions. To reduce the noise error, various techniques such as filtering or averaging can be used. Filtering techniques can help remove random variations in the sensor readings and provide more accurate measurements. Averaging can also be used to reduce noise by averaging out random variations in the measurements over time.\cite{Faragher:2014}
\end{enumerate}

\section{Affected Parameters}
Bias, drift, and noise errors will affect all the accelerometers and gyroscopes parameters. Here's how they impact each parameter:\newpage

\subsection{Accelerometer:}
\begin{itemize}
    \item Ax, Ay, and Az: The accelerometer measures linear acceleration in three directions, X, Y, and Z. If the accelerometer experiences bias, there will be a constant offset in the acceleration measurement in all three directions, even when there is no actual acceleration. 
    \item Drift in an accelerometer can cause a slow, gradual change in the measured acceleration values over time. This means that even when the device is stationary, the accelerometer readings may drift away from the true acceleration values. For example, if the accelerometer is used to measure the tilt angle of a platform, drift can cause the tilt angle to slowly change even when the platform is not moving. Over time, this can result in significant errors in the measured tilt angle.
    \item Noise in an accelerometer can cause random fluctuations in the measured acceleration values. This means that even when the device is stationary, the accelerometer readings may vary randomly around the true acceleration values. For example, if the accelerometer is used to measure the vibration of a machine, noise can cause the measured vibration amplitude to fluctuate randomly around the true amplitude. This random fluctuation can make it difficult to accurately measure the machine's vibration characteristics.
\end{itemize}

\subsection{Gyroscope:}
\begin{itemize}
    \item $\Omega$x, $\Omega$y, and $\Omega$z: The gyroscope measures the rate of change of angular velocity in three directions, Roll, Pitch, and Yaw. Bias in the gyroscope can cause a constant offset in the measured angular velocity values, even when there is no actual rotation. \cite{Noureldin:2013}
    \item Drift in gyroscopes is caused by mechanical imperfections in the device, temperature changes, or other external factors that can lead to a gradual change in the measured angular velocity value over time, even when the device is not rotating. The drift error can accumulate over time and can significantly affect the accuracy of the gyroscope measurements. For example, a drone that uses a gyroscope for stabilization can experience drift error over time, causing it to drift off course and potentially crash.
    \item Noise in gyroscopes is caused by random fluctuations in the measured angular velocity values due to external disturbances such as vibrations or electromagnetic interference. The noise error can affect the precision of the gyroscope measurements and can lead to instability in the application. For example, in robotics, noise error can cause the robot to deviate from its intended path or make inaccurate movements.
\end{itemize}

To minimize these errors, calibration and filtering techniques can be used. Calibration can remove bias by using known reference values to adjust the sensor's measurements. Zero-g and zero-rate calibration techniques can be used to remove bias from accelerometers and gyroscopes, respectively. Filtering techniques can be used to remove noise and reduce drift. For example, a Kalman filter can be used to estimate the true acceleration or angular velocity values based on previous measurements and sensor models.\cite{Lin:2005}



