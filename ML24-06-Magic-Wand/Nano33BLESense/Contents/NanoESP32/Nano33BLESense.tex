%%%%%%%%%%%%%%%
%
% $Autor: Wings $
% $Datum: 2020-01-29 07:55:27Z $
% $Pfad: Nano33BLESense.tex $
% $Version: 1785 $
%
%
%%%%%%%%%%%%%%%


%todo citatzions in correct manner
%source: {\tiny Quelle: \url{https://www.arduino.cc/pro/tutorials/portenta-h7}}



%todo GitLab\ToDo\TinyML
% TinyML Seite 222

%source: \url{https://www.reichelt.de/de/de/arduino-nano-33-ble-sense-nrf52840-ohne-header-ard-nano-33bles-p261306.html?PROVID=2788&gclid=Cj0KCQjwna2FBhDPARIsACAEc_Uui0sowk2McNvt69o9wzQJtqHWW2n5l_OHng2pSRcmfYqUsiosbdUaAp47EALw_wcB&&r=1}


\chapter{Arduino Nano 33 BLE Sense}


Arduino is an open-source electronics platform based on flexible, easy-to-use hardware and software. It provide us on board microcontroller and microprocessor kits for making digital and analogue devices. The microcontroller, oftenly called as tiny computer embed on the arduino board, normally in order to run these microcontroller we need some type of electronics e.g; diodes, resisters, capacitors, and transistors for making the voltage and current balancing. But the arduino team make a user friendly environment for getting rid of these electronics complication to run the hardware on software, just power the board as per the required voltage write the desired programm and upload it in a few seconds, it will bring everything on the board and  make us independent from any worry about the electronics complication. By the technological advancement in semiconductor and electronics  industry, the  control problems are now being solved by using these small size microcontroller instead of mechanical and electrical swithes. All Arduino boards have one thing in common which is a microcontroller, it is basically a really small computer, which help us to make a edge computing application.\cite{Arduino:2021b}
\Mynote{Advertisement! Rewrite more as an engineer!}

\section{Arduino Nano 33 BLE Sense}

The Arduino Nano Family is a set of boards with a tiny footprint, packed with features. It ranges from the inexpensive, entry model Nano , to the more feature-packed Nano BLE Sense / Nano RP2040 Connect which comprises of Bluetooth / Wi-Fi radio modules. These boards also have a set of embedded sensors, such as temperature, humidity, pressure, gesture, microphone and more. They can also be programmed with MicroPython and supports Machine Learning. \cite{Raj:2019}.

Arduino Nano 33 BLE Sense is one type of arduino family board, which is come up with Bluetooth Low Energy (BLE) capability for communication and set of sensors. This compact and reliable Nano board is built around the NINA B306 module for BLE and Bluetooth 5.0 communication; Bluetooth 5.0 is the latest version of the Bluetooth wireless communication standard. Use of microcontrollers has become inevitable in almost every field of engineering.  The Arduino Nano 33 BLE Sense module is based on Nordic nRF52480 processor that contains a powerful Cortex M4F and the board has a rich set of sensors that allow the creation of innovative and highly interactive designs. Its reduced power consumption, compared to other same size boards, together with the Nano form factor opens up a wide range of applications. 



The model name Arduino Nano 33 BLE Sense, itself puts out some important information. It is called ``Nano'' because of its compact nano size. ``33'' is included in the model name to indicate that the board operates on 3.3V.  Then the name ``BLE'' indicates that the module supports Bluetooth Low Energy and the name ``Sense'' indicates that it has on-board sensors like accelerometer, gyroscope, magnetometer, temperature and humidity sensor, Pressure sensor, Proximity sensor, Colour sensor, Gesture sensor, and even a built-in microphone. \cite{Raj:2019}.


Also, for making the RGB color and person detection, we need to make the interface of BLE 33 Sense with camera shield Arducam OV2640. The Arducam OV2640 camera use  to detect the RGB, object detetion and also the gesture too. Arduino Nano 33 BLE Sense and camera shield Arducamis a perfect match for making ML and AI application, by having the set of sensors on board we just need to install the respective library on Arduino boar and it support the funcnality of sensors and Machine learning application.
\Mynote{Advertisement! Rewrite more as an engineer!}


The following figure ~\ref{ArduinoNano33} shows a Arduino Nano 33 BLE Sense, shows how compact it is and small in size.

\begin{figure}[ht]
    \centering
    \includegraphics[width=0.45\linewidth]{Arduino/ArduinoNano33BLESense}
    
    \includegraphics[width=0.45\linewidth]{Arduino/ArduinoNano33BLESense2}
 
    \includegraphics[width=0.45\linewidth]{Nano33BLESense/ArduinoNano33BLESenseTopology}
 
    
    \caption{Arduino Nano 33 BLE Sense, see \href{https://store.arduino.cc/arduino-nano-33-ble-sense}{Arduino Store}} 
    \label{ArduinoNano33}
    \Mynote{tikz?}
\end{figure}


The BLE (Bluetooth Low Energy) compact and reliable Nano board are built on NINA B306 module for BLE and Bluetooth 5 communication; the NINA B306 module based on Nordic nRF52480 processor that contains a powerful Cortex M4F CPU, Its architecture fully compatible with Arduino IDE Online and Offline. The Arduino Nano BLE 33 Sense have a following set of sensors on board, ADPS-9960, LPS22HB, HTS221, LSM9DS1,and MP34DT05-A. It is small in size, having all the required sensor on board. \cite{Arduino:2021}


The Arduino Nano 33 BLE Sense have the following set of Sensors, BLE module and its functionality below.

\begin{itemize}
    \item The Bluetooth is managed by a NINA B306 module.
    \item The ADPS-9960 is a digital proximity, ambient light, RGB and gesture sensor.
    \item The LSM9DS1 is a system-in-package featuring a 3D digital linear acceleration sensor, a 3D digital angular rate sensor, and a 3D digital magnetic sensor.
    \item The LPS22HB reads barometric pressure and environmental temperature.
    \item The HTS221 senses relative humidity.
    \item The MP34DT05 is support the sound detection.
\end{itemize}


\section{Was ist der Unterschied zwischen Rev1 und Rev2?}

Es gab einige Änderungen an den Sensoren zwischen beiden Revisionen:

Austausch des IMU von LSM9DS1 (9 Achsen) durch eine Kombination aus zwei IMUs (BMI270 - 6 Achsen IMU und BMM150 - 3 Achsen IMU).
Austausch des Temperatur- und Feuchtigkeitssensors von HTS221 zu HS3003.
Austausch des Mikrofons von MP34DT05 zu MP34DT06JTR.
Zusätzlich wurden einige Komponenten und Änderungen vorgenommen, um die Benutzererfahrung zu verbessern:

Austausch der Stromversorgung von MPM3610 zu MP2322.
Hinzufügen eines VUSB-Lötpads auf der Oberseite der Platine.
Neue Testpunkte für USB, SWDIO und SWCLK.


Muss ich meinen Sketch für die vorherige Revision ändern?

Für Sketches, die mit Bibliotheken wie LSM9DS1 für das IMU oder HTS221 für den Temperatur- und Feuchtigkeitssensor erstellt wurden, müssen für die neue Revision diese Bibliotheken zu folgenden geändert werden: Arduino\_BMI270\_BMM150 für das neue kombinierte IMU und Arduino\_HS300x für den neuen Temperatur- und Feuchtigkeitssensor.


Rev 1
\HREF{https://downloads.arduino.cc/libraries/github.com/bcmi-labs/Arduino_TensorFlowLite-1.15.0-ALPHA.zip 3}{TensorFlow Lite lib}



\section{Unterschied zwischen dem Arduino Nano 33 BLE Sense Rev1, dem Arduino Nano 33 BLE Sense Rev2 und dem Arduino Nano 33 BLE Sense Lite}

Das Tiny Machine Learning Kit enthält nur den Arduino Nano 33 BLE Sense  Lite. Im Folgenden werden hier die Unterschiede der drei verschiedenen Versionen erläutert.  

\bigskip

Die Unterschiede zwischen dem Arduino Nano 33 BLE Sense und dem Arduino Nano 33 BLE Sense Rev2 betreffen die IMU, den Temperatur- und Feuchtigkeitssensor, das Mikrofon, den Crypto-Chip und den Spannungswandler. In der zweiten Revision wurde die IMU LSM9DS1 durch zwei Module ersetzt: einerseits durch einen Beschleunigungs- und Drehratensensor BMI270 und das Magentometer  BMI150. Bei den Feuchtigkeitssensoren wurde der HTS221 durch den HS3003 ausgetauscht, wobei letzterer eine höhere Genauigkeit verspricht. Die Werte der Mikrofone sind trotz des Wechsels von dem MP34DT05 zum MP34DT06JTR gleichgeblieben. Ebenso unterscheiden sich nur die Bauteilbezeichnungen der Spannungswandler. Bei der ersten Generation wird der MPM3610 verwendet, der Rev2 hat den MP2322 verbaut. Bei dem Rev2 ist allerdings der Crypto-Chip nicht mehr vorhanden.

\bigskip

Der Arduino Nano 33 BLE Sense Lite unterscheidet sich nur geringfügig von dem Arduino Nano 33 BLE Sense Rev2. Der einzige Unterschied zwischen den beiden Modellen ist, dass die Lite Version nicht über den HTS221, sondern über den Sensor LPS22HB verfügt. Hierbei  handelt es sich um einen Drucksensor, mit dem es allerdings auch möglich ist, die Temperatur zu messen. Feuchtigkeit lässt sich also nicht mit dem Lite messen. Um relative Luftfeuchtigkeit zu messen, muss ein separater Sensor angeschlossen werden. Der Grund für diese Entscheidung ist, dass die Firma Arduino Schwierigkeiten hat, den aktuellen Lagerbestand aufrechtzuerhalten. \cite{Filipi:2022}




\section{On-Board Sensor Description}

Arduino Nano 33 BLE sense come up with the set of embed sensor on the board. The available embed sensors are commonly use for measuring both the analog and digital values around the sorrounding. Arduino Nano 33 BLE sense is very small 45mm $\times$ 18mm in size, which  makes it very usefull for Internet of things (IOT) and Artificial intelligence (AI) application as a embed device where space is the main constrained issue. It is low power consumption board and operate normally on 3.3 V, we can say that this small size low power consumption board can operate on small batteries even for many months. Due to on-board available sensor, the low power consumption and mini architecture we can use this nano board anywhere. The Arduino Nano 33 BLE Sense is a completely new board on a well-known form factor. For getting detail information about each component of Arduino Nano 33 BLE Sense and data sheets of each sensor the following links give us a detail information \cite{Arduino:2021}. The short description of each sensor are as follow. 

\begin{itemize}
    \item The ADPS-9960 is a digital proximity, ambient light, RGB and gesture sensor. it can measure the proximity distance, light, color and gestures when moving close with the borad.
    \item The sensor LSM9DS1 is a 9 axis \ac{imu} use as a accelerometre, gyroscope, and magnatometre, this 9 axis sensor is ideal for wearable devices.
    \item The sensor LPS22HB is a barometric pressure sensor, it measures the environmental pressure which is usefull for simple weather station monitoring. 
    \item The sensor HTS221 senses the relative humidity, and temperature, to get highly accurate measurements of the environmental conditions.
    \item The sensor MP34DT05 is the digital microphone. it is usefull for capturing, analyzing and detecting the sound in real time.
    \item The USB port allows you to connect  Arduino Nano 33 BLE sense to your machine.
    \item There are 3 different LEDs that can be accessed on the Nano BLE Sense: RGB Programmable LED , the built-in  orange Programmable LED and the Power LED.
    
\end{itemize}

The below figure \ref{ArduinoNano33BLESenseArchitecture} show the embed sensors on the board, with powerful processor as compared to other arduino boards the nRF52840 from Nordic Semiconductors, a 32-bit ARM\textsuperscript{\textregistered} Cortex\textsuperscript{\texttrademark}-M4 CPU processor running at 64 MHz are as follow.

\begin{figure}[ht]
    \centering
    \includegraphics[width=\linewidth]{Nano33BLESense/ArduinoNano33.jpg}
    \caption{\textbf{Components in Arduino Nano 33 BLE Sense \cite{Raj:2019}}}
%    \includegraphics[width=0.5\linewidth]{Nano33BLESense/NANO-33-BLE-Sense_sensor-indentification}
    \label{ArduinoNano33BLESenseArchitecture}
    \Mynote{Overpic}
\end{figure}

Another point to bear in mind is the overall 'trueness' of sensor readings based on where do you place the actual sensors in the environment, as this could be quite critical. The operating temperature should not exceed 85$^0$C and not lower than -40$^0$C. Other factors like humidity level and air pressure values should also be kept in check.

\begin{table}
   \begin{center}
            \begin{tabular}{llm{90mm}} 
                \textbf{\MapleCommand{MICROCONTROLLER}}  & nRF52840\\
                \textbf{\MapleCommand{OPERATING VOLTAGE}}  & 3.3V\\
                \textbf{\MapleCommand{INPUT VOLTAGE (LIMIT)}}  & 21V \\
                \textbf{\MapleCommand{DC CURRENT PER I/O PIN}}  & 15 mA \\
                \textbf{\MapleCommand{CLOCK SPEED}} & 64MHz \\
                \textbf{\MapleCommand{CPU FLASH MEMORY}}  & 1MB  \\
                \textbf{\MapleCommand{SRAM}}  & 256KB  \\
                \textbf{\MapleCommand{LED\_BUILTIN}}  & 13 \\
                \textbf{\MapleCommand{IMU (Accelerometer, Gyroscope, Magnetometer)}}  & LSM9DS1 \\
                \textbf{\MapleCommand{MICROPHONE}}  & MP34DT05 \\
                \textbf{\MapleCommand {GESTURE, LIGHT, PROXIMITY, COLOUR}}  & APDS9960 \\
                \textbf{\MapleCommand{BAROMETRIC PRESSURE}}  & LPS22HB \\
                \textbf{\MapleCommand{TEMPERATURE, HUMIDITY}}  & HTS221 \\	
            \end{tabular}
        \end{center}
    \caption{Technical Specifications of Arduino Nano 33 BLE Sense \cite{Arduino:2021}}
\end{table}


\subsection{Gesture, Proximity, and Color Detection Sensor ADPS-9960}

The APDS-9960 device features advanced Gesture detection, Proximity detection, Digital Ambient Light Sense (ALS) and Color Sense (RGB). \cite{Arduino:2021} Gesture detection utilizes four directional photodiodes to sense reflected IR energy (sourced by the integrated LED) to convert physical motion information (i.e. velocity, direction and distance) to a digital information.

For the following Applications the sensor is in use:

\begin{itemize}
    \item Gesture Detection
    \item Color Sense
    \item Ambient Light Sensing
    \item Proximity Sensing
\end{itemize}

\subsection{Accelerometer, Gyroscope, and Magnetometre Sensor LSM9DS1}

The LSM9DS1 is a system-in-package featuring a 3D digital linear acceleration sensor, a 3D digital angular rate sensor, and a 3D digital magnetic sensor. \ac{imu}'s work by detecting rotational movements across the 3 axis known as Pitch, Roll and Yaw. To achieve the same, it depends on Accelerometer, Gyroscope and Magnetometer. The accelerometer gives the velocity at which the \ac{imu} module moves. The gyroscope measure the rotational movement rate on the \ac{imu}. Magnetometer measures the force of gravity acting on the \ac{imu}.

For the following Applications the \ac{imu} is in use:


\begin{itemize}
  \item Indoor navigation
  \item Advanced gesture recognition
  \item Gaming and virtual reality input devices
  \item Display/map orientation and browsing
  \item Consumer electronics; Smartphones, tablets, fitness trackers for motion sensing and orientation.
  \item Compact transportation solutions like Segway.
  \item Sports Technology - helping athletes to know how they can improve their movements.
\end{itemize}


There are also certain disadvantages of using the \ac{imu} and points that need to be kept in mind while using an \ac{imu} sensor. Accumulated error or \textit{'Drift'} is the main disadvantage of \ac{imu}s, present due to its constant measuring of changes and rounding off its calculated values off. When such a process happens for a prolonged period of time, it can lead to significant errors. The best way to avoid the \textit{drift} factor is to use a good quality \ac{imu} Sensor and make sure the \ac{imu} sensor is calibrated. \cite{STMicroelectronics:2015}
\Mynote{The explanation of drift is to small; citations}

\subsubsection{Calibration of an IMU}

\Mynote{What is calibration? citations!}

On research it was found that there are various methods to calibrate the sensors involved, the time period between each calibration is also not defined specifically, however it is advised that regular calibration is done especially when there are strange outputs noticed. Few methods of calibration are briefed below:

\subsubsection{Low and High Limit Method}

In this method the sensor is rotated in circles along each axis a few times. The midpoint is then found between the two extremes. If there is no offset, the midpoint is close to zero, but if there is a slight deviation from zero, this figure is the hard iron offset, which is the result of the distortion caused by the Earth's magnetic field. This method is mainly used to calibrate the Magnetometer \cite{Mallon:2015}

\subsubsection{Magneto V1.2}

In this method, the raw magnetometer data is pre-processed with axis specific gain correction to convert the raw output into nanoTesla: 

\begin{center} 
    
    \PYTHON{Xm\_nanoTesla = rawCompass.m.x*(100000.0/1100.0);}
    
    \PYTHON{Gain X [LSB/Gauss] for selected input field range}
    
    \PYTHON{Ym\_nanoTesla = rawCompass.m.y*(100000.0/1100.0);}
    
    \PYTHON{Zm\_nanoTesla = rawCompass.m.z*(100000.0/980.0);}
    
    
\end{center} 

This converted data is saved into the file  \FILE{Mag\_raw.txt} that you open with the Magneto program. To start using this method, we first need to replace the (100000.0/1100.0) scaling factors with values that convert your specific sensors output into nanoTesla. Rather than simply finding an offset and scale factor for each axis, Magneto creates twelve different calibration values that correct for a whole set of errors: bias, hard iron, scale factor, soft iron and misalignment.

A side benefit of this is that it can be used to calibrate accelerometers as well. You might again need to pre-process your specific raw accelerometer output, taking into account the bit depth and G sensitivity, to convert the data into milliGalileo. Then enter a value of 1000 milliGalileo as the ``norm'' for the gravitational field. \cite{Mallon:2015}




\Mynote{Here, we need more information because we want to use it:
\begin{itemize}
    \item General infomration about imus
    \item angle to roation matrix with example!
    \item calibration
    \item drift
    \item $\ldots$
    \item specials for this imu
    \item description of the parameters for coding
    \item example code, see Code/Nano33BLESense/IMU/Arduino
\end{itemize}
}

\subsection{Pressure Sensor LPS22HB}
The LPS22HB is an ultra-compact piezoresistive absolute pressure sensor which functions as a
digital output barometer.

For the following Applications the sensor is in use:



\begin{itemize}
    \item Altimeters and barometers for portable devices 
    \item Weather station equipment
    \item Sports watchs
\end{itemize}

A sensor element is installed as well as an IC interface that communicates via an I\textsuperscript{2}C or SPI bus. The function is given in a temperature range from $-40^\circ C$ to $+85^\circ C$. [STM17]

\begin{itemize}
  \item Absolutdruckbereich: 260 bis 1260 hPa
  \item Versorgungsspannung: 1,7 bis 3,6 Volt
  \item 24-bit Druckdatenausgabe
  \item 16-bit Temperaturdatenausgabe
\end{itemize}

\subsection{Relative Humidity and Temperature Sensor HTS221}
The HTS221 is an ultra-compact sensor for relative humidity and temperature. It includes a sensing element and a mixed signal to provide the measurement information through digital serial interfaces.

For the following Applications the sensor is in use:


\begin{itemize}
    \item Air conditioning, heating and ventilation 
    \item Air humidifiers
    \item Refrigerators
    \item Smart home automation
    \item Industrial automation
\end{itemize}  

Dieser Sensor kommuniziert über den I\textsuperscript{2}C- und SPI-Bus. Dieser Sensor ist einsetzbar in einem Temperaturbereich
von $-40^\circ C$ bis $+120^\circ C$. Zur Spannungsversorgung werden 1,7 bis 3,3 Volt benötigt. Eine Temperaturmessung erfolgt mit einer Genauigkeit von $\pm 5^\circ C$. [STM23]

\begin{itemize}
    \item GND - Ground
    \item DRDY - Data ready output signal
    \item SCL/SPC - I2C serial clocl (SCL) \& SPI serial port clock (SPC)
    \item VDD - Stromversorgung
    \item SDA/SDI/SDO - I\textsuperscript{2}C serial Data (SDA) \& 3 wire-SPI serial data input/output (SDI/SDO)
    \item SPI enable - I\textsuperscript{2}C/SPI mode selection
\end{itemize}

\subsection{Digital Microphone MP34DT05-A}

The MP34DT05-A is an ultra-compact, low-power, omni directional, digital microphone built with a capacitive sensing element and an IC interface. The sensing element, capable of detecting acoustic waves, is manufactured using a specialized silicon micromachining process dedicated to producing audio sensors.
The MP34DT05 is a low-distortion microphone with a signal-to-noise ratio of 64 dB. The sensitivity is -26 dBFS $\pm$ 3 dB. The \ac{aop} is 122.5 dBSPL.

The output signal of the microphone is a PDM signal. This signal is a binary signal modulated by \ac{pdm}  from the analogue signal. \cite{STMicroelectronics:2021}

For the following Applications the sensor is in use:

\begin{itemize}
    \item Speech recognition 
    \item Portable media player
    \item Mobile Terminal
\end{itemize}

\begin{figure}[h]
  \includegraphics[width=16cm]{Microphon/MikrophonCP}
  \caption[Circuit diagram microphone]{Circuit diagram microphone \cite{STMicroelectronics:2021}}
\end{figure}

The Arduino nano 33 BLE Sense has a built-in microphone that uses PDM (Pulse-Density Modulation) to convert sound into digital data. You can use the PDM library to access the microphone data and perform various tasks with it. Here is a simple example of using the builtin microphone for an Arduino nano 33 BLE Sense:

\begin{code}
    \begin{Arduino}
        // Include the PDM library
        #include <PDM.h>
        
        // Buffer to store the microphone data
        short sampleBuffer[256];
        
        // Variable to store the sound level
        int soundLevel = 0;
        
        // Callback function for PDM data
        void onPDMdata() {
            // Read the PDM data
            int bytesAvailable = PDM.available();
            PDM.read(sampleBuffer, bytesAvailable);
            
            // Calculate the sound level
            soundLevel = 0;
            for (int i = 0; i < bytesAvailable / 2; i++) {
                soundLevel += abs(sampleBuffer[i]);
            }
            soundLevel /= bytesAvailable / 2;
        }
        
        void setup() {
            // Initialize serial communication
            Serial.begin(9600);
            while (!Serial);
            
            // Initialize PDM with a sample rate of 16 kHz and 16-bit resolution
            PDM.begin(1, 16000);
            PDM.onReceive(onPDMdata);
        }
        
        void loop() {
            // Print the sound level to the serial monitor
            Serial.println(soundLevel);
            delay(100);
        }
    \end{Arduino}
    \caption{Simple example using of the builtin microphone of the Arduino Nano 33 BLE Sense}\label{code:microphone}
\end{code}


The code~\ref{code:microphone} will print the sound level of the microphone to the serial monitor every 100 milliseconds. 

You can modify this code to perform other tasks with the microphone data, such as controlling LEDs, playing sounds, or sending data to other devices. For more information and examples, you can check out the PDM library documentation or the Arduino nano 33 BLE Sense page. 

\Mynote{How to install the library PDM\\description of the lib\\ functions}


\subsection{Bluetooth Module nRF52840}

The nRF52840 is an advanced, highly flexible single chip solution for today’s increasingly demanding Ultra Low Power (ULP) wireless applications for connected devices on our person, connected living environments and the \ac{iot} at large. It is designed ready for the major feature
advancements of Bluetooth 5 and takes advantage of Bluetooth 5's increased performance capabilities. \cite{Arduino:2021b}

\textbf{Applications}

\begin{itemize}
    \item Smart Home products
    \item Industrial mesh networks
    \item Smart city infrastructure
    \item Connected watches
    \item Advanced personal fitness devices
    \item Wearables with wireless payment
    \item Connected Health
\end{itemize}


\section{Arduino Nano 33 BLE Pin Configuration}

Arduino Nano 33 BLE is an advanced version of Arduino Nano board that is based on a powerful processor the nRF52840. The figure ~\ref{Schnittstellen} shows that the board has the following pin configuration. \href{https://www.etechnophiles.com/arduino-nano-33-ble-sense-pinout-introduction-specifications/}{Pin Configuration}


\begin{description}
  \item[Digital pin] The number of digital I/O pins are 14 which receive only two values HIGH or LOW. These pins can either be used as an input or output based on the requirement. When these pins receive 5V, they are in a HIGH state and when they receive 0V they are in a LOW state.
  \item[Analog pin] Total 8 analog pins available on the board A0 -- A7. These pins get any value as opposed to digital pins that only receive two values HIGH or LOW. These pins are used to measure the analog voltage ranging between 0 to 5V.
  \item[PWM pin] All digital pins can be used as PWM pins. These pins generate analog results with digital means.
  \item[SPI pin] The board supports serial peripheral interface (SPI) communication protocol. This protocol is employed to develop communication between a controller and other peripheral devices like shift registers and sensors. Two pins are used for SPI communication i.e. Master Input Slave Output (MISO) and Master Output Slave Input (MOSI) are used for SPI communication. These pins are used to send or receive data by the controller.
  \item[I2C pin] The board carries the I2C communication protocol which is a two-wire protocol. It comes with two pins SDL and SCL.
  \item[UART pin] The board features a UART communication protocol that is used for serial communication and carries two pins Rx and Tx. The Rx is a receiving pin used to receive the serial data while Tx is a transmission pin used to transmit the serial data.
  \item[External Interrupts pin] All digital pins can be used as external interrupts. This feature is used in case of emergency to interrupt the main running program with the inclusion of important instructions at that point.
  \item[LED at Pin 13 and AREF pin] There is an LED connected to pin 13 of the board. And AREF is a pin used as a reference voltage for the input voltage.
\end{description}



\begin{figure}[ht]
    \centering
    \includegraphics[width=0.5\linewidth]{Nano33BLESense/Schnittstellen}
    \caption{Arduino Nano 33 BLE Pin Configuration}
    \label{Schnittstellen}
\end{figure}


\section{Was fehlt}

\begin{itemize}
  \item Beschreibung der Sensoren
    \begin{itemize}
      \item Hintergrundwissen
      \item Beschreibung
      \item Eigenschaften
      \item Kalibrierung
    \end{itemize}
  \item Beschreibung der Bibliothek
    \begin{itemize}
      \item Beschreibung
      \item Installation
      \item Funktionen
    \end{itemize}
  \item Laufzeit, Speicherplatz, \ldots
  \item Software-Dokumentation
  \item Verwendung von Werkzeugen, z.B. Doxygen
    \begin{itemize}
      \item Beschreibung, inklusive im Quellcode; z.B. Header
      \item Handbuch
      \item Ablaufdiagramm
      \item \ldots
\end{itemize}
\item Interpretation der Ergebnisse
\item Literatur, Bilder
\item \ldots
\end{itemize}


